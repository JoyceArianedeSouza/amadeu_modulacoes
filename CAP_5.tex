\chapter{Da biopolítica à modulação: psicologia social e algoritmos como agentes da assimilação neoliberal}

\begin{flushright}
\emph{Cínthia Monteiro}
\end{flushright}

Em ``O Nascimento da Biopolitica'', Foucault (1999) não procura
encontrar a história do que é verdadeiro ou falso, mas a história da
veridição, ou seja, como as verdades nos mecanismos de controle foram
construídas e assimiladas pela sociedade. Contudo a biopolítica se
diverge do biopoder, uma vez que a primeira representa a construção da
outra, ou sua estrutura edificante. A biopolítica representa o discurso
que constrói o biopoder, e uma vez solidificado, o bipoder é o que
garante uma submissão carregada de legitimidade, ou mesmo o poder de
submeter sujeitos, povos, entre outros (Foucault, 1999).

O poder de vida e morte sai das mãos do soberano e vai para uma espécie
de enquadramento nos moldes de sujeito ideal do sistema neoliberal. Os
mecanismos de \emph{fazer viver e deixar morrer} ganham uma qualidade
técnica de quase ``soberania da ciência'', com destaque para correntes
positivistas\footnote{Positivismo: Corrente filosófica francesa que
  surgiu no começo do século XIX. O principal idealizador do positivismo
  foi o filósofo Augusto Comte (Trindade, 2007)}. Foucault destaca o
racismo de Estado como mecanismo de domínio das populações que deveriam
se submeter as populações superiores. É nesse aspecto que a biopolítica
se torna um mecanismo científico para enquadrar e submeter os
indivíduos. Uma tecnologia disciplinar que não visa somente moldar os
corpos, mas os seres humanos enquanto espécie. Políticas
segregacionistas, raciais, exclusão de doentes mentais foram centrais no
século XIX e XX, onde a ideologia de criação do ser humano perfeito
submeteram, em nome do progresso social, diversas pessoas a um
ideal de sociedade. Foucault utiliza o exemplo de prisões e hospícios
como concretização dos ideais de segregação dos indesejáveis.

A medicina tem papel fundamental na construção da biopolítica e do
biopoder, pois além do poder de vida e morte, as políticas de submissão
popular ao sanitarismo entre outras políticas públicas de intervenção
técnica tem forte poder na transformação do ideário social e controle.
Trata-se de um ramo científico com propriedades para influir sobre os
corpos da população com grande peso de legitimidade, que se exerce sobre
vida e morte e não necessariamente o poder do soberano de assassínio.
Quando falamos sobre poder de vida e morte, tratamos de deixar viver ou
morrer ou do assassínio indireto que consiste em exclusão, exposição ao
risco de morte, aumento das chances de determinado grupo social morrer
ou a exclusão política, rejeição que encaminha pessoas para a
marginalidade (Foucault, 1999). São políticas de controle de natalidade,
mortalidade, incapacidades biológicas e a limitação desses indivíduos no
âmbito social, com as quais se passa a tratar da ``população como
problema político, como problema ao mesmo tempo científico e político,
como problema biológico e como problema de poder''( 293).A medicina como
forte instrumento da biopolítica desenha o projeto de sucesso e de
fracasso que leva as pessoas a se enquadrarem nesse modelo ou serem
relegadas a marginalidade. Assim como o liberalismo desenha padrões de
sucesso e ideais a serem atingidos, modelos assimilados pelas pessoas
como triunfais, bem como papéis adequados de destaque, tanto essas
posições almejadas quanto a busca por elas exercem a força do biopoder
sobre as pessoas. A submissão a padrões estéticos, de conduta, de
consumo e dinâmicas exaustivas de trabalho seguem a premissa de um dia
alcançar um determinado papel social e simbólico de sucesso. A
obediência, como já esclarecida por Foucault (1999), se ampara da promessa
de sucesso econômico e ascensão por meio desses sacrifícios. Esclarecer
para aqueles que já assimilaram os valores da biopolítica que não vai
acontecer o que prega a ideologia, ou que é um sistema insustentável, é
uma tarefa árdua, pois os mecanismos de subordinação não são facilmente
visíveis ou identificáveis.

Somando-se a esses fatores, uma mudança complexa e de mesma faceta não
facilmente observável no sistema de produção de bens de consumo, embora
de extrema relevância, está o enfoque no valor do imaterial\footnote{Imaterial:
  valor imaterial corresponde ao valor de troca, relacionado ao valor
  que não pode ser quantificável por itens ou quantidade de trabalho
  empregado, uma vez que diz respeito à construção de identidades,
  projeções de personalidade(GORZ, 2005).}, cuja importância cada vez
mais se sobressai ao da produção em valor numérico (Gorz, 2005). Segundo
André Gorz, é fundamental para a compreensão da economia moderna e
composição técnica do capital entender o sistema de produção e a
dimensão imaterial dos produtos. O valor dos bens de consumo na atual
economia global está cada vez mais subordinado a fatores e bens
imateriais, ou seja, não se relaciona a utilidade, valor de produção,
números de produção, mas a uma construção identitária relacionada aos
itens de consumo.

Tendo isso em vista, a análise da ``composição técnica do capital'' é
fundamental para a compreensão dos mecanismos de condicionamento, uma
vez que isse implica diretamente no entendimento da composição técnica
dos sistemas de trabalho, determinando quem produz, o que produz e como
produz na economia global atual (Hardt e Negri, 2009). A exploração do
trabalho desenvolve uma nova faceta quando além de gerar riqueza através
da força de produção do trabalho, também relaciona seus ``produtos'' com
a identidade do trabalhador, incluindo uma dinâmica de exploração
biopolítica do trabalho.

A junção da biopolítica com os valores imateriais de produção insere
outra questão a ser analisada, uma vez que aborda um novo tipo de padrão
na sociedade, naturalizando um arquétipo de características que os
indivíduos devem ter relacionadas a ideais de valores pertencentes a
essa nova dinâmica do capitalismo (Hardt e Negri , 2009). Essa análise
se assemelha à perspectiva marxista de consumo fetichista\footnote{Fetichismo:
  é a percepção das relações sociais envolvidas na produção, não como
  relações entre as pessoas, mas como as relações econômicas entre o
  dinheiro e as commodities negociadas no mercado (MARX, 1988).},
somente em partes, pois os produtos da biopolítica excedem essa corrente
de pensamento sobre a produção de bens materiais. A nova dinâmica de
valoração transpõe a mensuração quantitativa econômica por sua
característica de subjetividade, visto que o valor do imaterial está
ligado a fatores de construção ideológica nos próprios atores inseridos
nesse novo modelo de exploração.

O processo de acumulação capitalista se deve cada vez menos a exploração
da força de trabalho e produção, é mais externa a esses processos e cada
vez mais orgânica do ponto de vista do contexto biopolítico e imaterial,
se utilizando de formas subjetivas de exploração, tornando menos visível
a luta de classes, uma vez que constrói novos perfis de identidade a
serem almejados pela classe trabalhadora (Hardt e Negri , 2009).

Para além de itens de consumo, identidade relacionada ao consumo, é
criado um novo modelo de indivíduo na sociedade capitalista, uma espécie
reformulada de \emph{``Self made man}\footnote{Self made man: A
  expressão americana~\emph{Self Made Man}, expressão máxima do
  capitalismo moderno, corresponde ao homem que conseguiu sucesso por si
  mesmo, por seus próprios esforços e sua própria dedicação, traduzindo
  para o português, seria aquele homem que ``se fez''. Através dela é
  defendido o enriquecimento do homem moderno, e através dela se
  estrutura toda a sociedade atual(Ward, 1954).}\emph{'',} que não
depende nem mesmo de salário, é um ``empresário da sua força de
trabalho'' que providencia seu próprio status, através da sua formação,
busca por conhecimento, empreendedorismo, chamado por André Gorz de ``Eu
S/A''.

A interpretação de Gilles Deleuze (1992) sobre os novos mecanismos de
poder indica que o indivíduo por mais que seja submetido a esses
mecanismos, o cobiça, pois quer ter domínio daquilo que o controla,
constrói sua identidade relacionada ao imaterial e deseja consumir o que
o identifica dentro do âmbito social. O controle é feito por meio da
normalização, da naturalização desses mecanismos de poder, ao passo que
modela a existência de cada indivíduo e os processos de subjetivação. A
diminuição dos aparelhos repressivos de Estado, se deve ao aumento de
direcionamento das vontades individuais para satisfazer os desejos e
prazeres dos indivíduos (Foucault, 1997).

Para construir o poder disciplinar os indivíduos não são destruídos,
mas sim fabricados. Segundo Foucault, o capitalismo não se sustentaria
com base apenas na imposição e repressão. Os valores, o conhecimento,
são produtores de individualidades que naturalizam o poder nas intenções
e desejos dos indivíduos.

\begin{quote}
A verdade é deste mundo; Ela é produzida nele graças às múltiplas
coerções e nele produz efeitos regulamentados d poder. Cada sociedade
tem seu regime de verdade , sua ``política geral'' de verdade
(Foucault1997,pg.12).
\end{quote}

O poder disciplinar toma novo formato e estruturas no neoliberalismo. A
legitimidade científica que sustenta a força do biopoder se transmuta em
algo menos notável e mais poderoso justamente por esse motivo.

Para Deleuze, a submissão aos mecanismos disciplinares, ainda que fossem
legitimados pela ciência, não se deu de maneira passiva ou pacífica,
como diversos eventos históricos apontam e a exemplo da revolta da
vacina no Brasil e as diversas manifestações e protestos contra leis de
segregação racial nos Estados Unidos. Por esse motivo, fez-se a
transmutação do biopoder para algo muito mais intrínseco do que a
imposição respaldada pela legitimidade científica, ou seja, para algo
dotado de ``poderes'' psicológicos de manipulação e assimilação. André
Gorz aponta os valores imateriais dos produtos na lógica neoliberal de
consumo capitalista, uma vez que os indivíduos criam identidades junto
àquilo que consomem, enquanto Deleuze destaca o mesmo acontecendo com os
mecanismos de controle. A psicologia, assim como fora para a medicina, a
história, entre outras ciências, passa a ter papel central na edificação
das identidades submissas à lógica neoliberal.

Porquanto, o biopoder criou um padrão do que era ``normal'' e o que
precisava ser enquadrado na ``normalidade'', a psicologia social e a
psiquiatria adquirem um novo papel de controle social que enquadra todos
e seus diversos modos de existência. A pós-modernidade cria diferentes
modos de existência, subjetividades singulares, e a diversidade de
maneira geral, logo o controle pela maneira opressiva de padronização e
enquadramento não faz mais sentido. Para Deleuze e Guatarri (1972), os
novos mecanismos de controle não operam mais na lógica binária de
opressor/oprimido, mas agem de maneira quase invisível e molecular
embrenhando-se no meio social. As patologias tomam o cotidiano de todos,
o stress, a insônia, a depressão e os conflitos fazem parte da vida de
todos e ao mesmo tempo que todos são ``normais'', todos precisam do
tratamento, o que torna a intervenção psicossocial difícil de ser notada
e identificada.

É difícil identificar o ponto de incidência do poder a ser combatido,
uma vez que esse é assimilado e dissipado no meio das interações e vida
social, se esvanecendo entre as inúmeras experiências cotidianas (Rauter
e de Castro Peixoto,2009). A modulação é, portanto, esse mecanismo que
sujeita os indivíduos sem necessariamente utilizar da sujeição física e
normativa do corpo vistas no biopoder descrito por Foucault. O
Panoptismo é assimilado pelos próprios indivíduos, de maneira que esses
fazem seu autocontrole dentro das normas difundidas nas subjetividades
(Rauter e de Castro Peixoto,2009). O ideal de indivíduo bem-sucedido em
uma sociedade produtiva é difundido por esses mecanismos e assimilado
como projeto a ser alcançado ou se enquadrar dentro do mesmo, levando
esse mecanismo de controle a níveis de sincronicidade com o modelo
liberal que dispensa a vigilância. O sistema de produção capitalista
passa a ser produtor de subjetividades por meio do que Deleuze e
Guatarri chamam de \emph{axiomática do capital}\footnote{Axiomática do
  Capital: capitalismo não opera por códigos, mas por um sistema de
  codificação e sobrecodificação das condutas (DeleuzeeGuatarri,
  1972/2010 apud Uhng Hur, 2012).}. A essa axiomática se confere a
naturalização de lógicas capitalistas que não mais precisam ser
impostas, pois são incontestáveis e o ``caminho certo'' para a obtenção
de capital e construção das individualidades. A meta de se enquadrar de
cada indivíduo é a própria vigilância.

O filósofo da tecnologia Yuk Hui (2015) trabalha as obras de Gilles
Deleuze e Gilbert Simondon para explicar o processo de modulação e a
transmutação da sociedade disciplinar para uma espécie de controle
essencial, em que o indivíduo assimila o controle como parte da sua
identidade. Segundo o autor, a sujeição dos corpos seria uma espécie de
``moldagem'', em que se imprime forçosamente a correção. Já a modulação
de Deleuze consiste na mudança da imposição para autorregulação dos
indivíduos. É possível, portanto, observar a passagem para uma nova
forma de operação que não mais consiste em restrição física e
enclausuramento. É um controle não explícito que não se impõe com
violência ou força nos indivíduos. Se a moldagem era um ``entalhe''
físico da forma humana no modelo de perfeição, a modulação transforma os
indivíduos em um molde autodeformável que pode ser continuamente
modificado de acordo com as demandas mercadológicas e sociais (Hui,
2015). O controle por modulação por não ser explicito faz com que o
indivíduo não enxergue a lógica capitalista que está seguindo e tome por
verdade algo que não é real, como a autonomia do suposto
autoempreendedorismo, em que a precarização do trabalho recebe uma
``glamourização'' neoliberal e um vendedor de hot-dog torna-se um
microempresário e seu carrinho é transformado em um food-truck. O
controle por modulação dá ao trabalhador uma suposta liberdade de
gerenciar sua vida e seu tempo, que na realidade não existe (Philipe
Zarifian apud Yuk Hui, 2015). Tanto no trabalho formal quanto no
informal, a ideia de \emph{``trabalhe quando quiser'',} \emph{``horas
flexíveis'',} mascaram trabalhos intermináveis daquele que deseja ser um
vencedor nesse sistema, daqueles que \emph{``vestem a camisa''} e saem
da \emph{``zona de conforto''}. Uma recente analogia foi feita em um
vídeo\footnote{Disponível em: \emph{https://www.youtube.com/watch?v=8EdjPneDOps}.
  Acesso em: setembro de 2018.} do Youtube, em que uma \emph{``Coach'',}
incentivadora do autoempreendedorismo relata como os judeus foram parar
em campos de concentração por não saírem da tal zona de conforto. O
vídeo foi imensamente criticado, com toda a razão, mas sem a devida
crítica sobre o termo \emph{``zona de conforto''}, que até poderia ser
relacionada com a subserviência, mas está presente na modulação e é
inerente ao sistema neoliberal. Substituindo é claro o \emph{``trabalhe}
\emph{quando quiser''} por \emph{``trabalhe o tempo todo''}.

A análise de Simondon é acrescida e, muitas vezes, utilizada pelo
próprio Deleuze para corroborar o processo operativo da modulação. Se
tornar, para os autores, não deveria ser diferente de ser, já que a
modulação é um processo contínuo de construção e justamente moldável de
individuação (Hui, 2015). A modulação é uma moldagem definitiva,
porém ajustável as mutáveis demandas do neoliberalismo.

Simondon (apud Hui, 2015) compara a modulação na psicologia social
aos sistemas tecnológicos que originaram o termo. A modulação, para o
autor, é tal qual um sistema de amplificação eletrônico, embora a
definição de amplificação seja transportada para os sistemas sociais.
Ainda que a falta de regulações rígidas ocasione uma aparente liberdade
de ação, os movimentos são antecipados por sistemas regulatórios, e
mesmo os atos livres são modulados de uma maneira quase autorregulatória
(Hui, 2015). Dentro desse sistema regulatório existem etapas que
concentram os mecanismos de modulação, e consistem primeiramente em um
reconhecimento e entendimento de padrões reproduzidos em um grupo
social, posteriormente a esse reconhecimento e por meio desse, a
antecipação de atividades é possibilitada. Ou seja, uma vez sumarizados
os padrões de comportamento, é possível tê-los registrados e dessa forma
antever a reação de um grupo ou sociedade. Por fim, a transferência
dessa responsabilidade de autorregulação para os indivíduos inseridos
nesse sistema, que consiste nos ideais de autoempreendedorismo, sucesso
no neoliberalismo (David Savat, apud Yuk Hui, pg.84, 2015).

A repercussão da modulação não atua somente por meio da amplificação
modulativa descrita por Simondon, tampouco ecoa apenas por meio das
redes, mas as etapas do processo de modulação citadas acima se
reproduzem e trabalham suportadas por uma série de aparatos que o
reconectam com sua origem tecnológica. Os padrões de reconhecimento e
antecipação de atividades dos usuários são ferramentas de uma espécie de
``behaviorismo de dados'', sendo seus instrumentos a
digitização\footnote{Digitização: transformação de processos e
  ferramentas, documentos e recursos para a forma digital da informática
  (LARIVIÈRE, 2016).} que toma conta dos mais diversos tipos de
instituições, tornando a operação de controle de dados por meio de
algorítmos\footnote{Algorítmos: conjunto das regras e procedimentos
  lógicos perfeitamente definidos que levam à solução de um problema em
  um número finito de etapas (CORMEN,T.H., LEISERSON, C.E., RIVEST,
  R.L., STEIN, C. 2009)} uma das formas centrais para se obter
informações por governos e demais estruturas sociais e governamentais
(Hui, 2015).

Assim, padrões de comportamento são desenhados, monitorados e
registrados para serem utilizados na influência e interferência social.
Esses dados coletados servem como mecanismo de fragmentação das
subjetividades ou como destacado por Antoinette Rouvroy (apud Hui,
pg.85, 2015) para processo de desubjetificação, em que o objeto é
fragmentado não podendo assim, manter uma identidade individual
coerente. Esse processo é característico do modo neoliberal de ação e
atinge em larga escala as interações sociais. Portanto o neoliberalismo
utiliza desses mecanismos para produzir \emph{hipersujeitos} capazes de
encaminhar o projeto de indivíduo de sucesso, que busca e cria as
próprias oportunidades, criador de si mesmo, apaixonado pela lógica
neoliberal de autoempreendedorismo, autocontrole e autoavaliação. A
modulação se expressa tanto nos níveis de assimilação e autorregulação
como nos mecanismos que garantem sua perpetuação, ou seja, indicando
mudanças nos padrões sociais que permitem os novos ajustes e
interpretações.

A lógica dos algoritmos, do calculável dentro da psicologia social, é
uma maneira de dominar a mente através de valores numéricos e padrões.
Tornar material aquilo que está interiorizado nos indivíduos, o
invisível em visível através de números e resultados quantitativos. A
partir disso se desenham projetos para administrar os indivíduos através
de esquematizações, separação por aptidões e capacidades psicológicas.
Sujeitar à mesma lógica capitalista, distribuir pessoas em diversos
papéis sociais de acordo com a meritocracia\footnote{Meritocracia:
  consiste em um neologismo cunhado pelo sociólogo britânico Michael
  Young nos anos 1950. Estabelece uma ligação direta entre mérito e
  poder. Pode ser entendida como um princípio de justiça, às vezes
  qualificado como utópico; mas pode também ser considerada como um
  instrumento ideológico que permite legitimar a desigualdade dentro de
  um sistema político.}, dentro das mesmas distinções de classe e papéis
disponíveis no mercado. É nesse espaço que entram as noções de liberdade
individual, enquanto as práticas de enquadramento procuram
silenciosamente moldar, transformar, classificar e reformar os
indivíduos (Rose, 2008).

Instrumentos como Facebook, entre outros websites, são capazes de
armazenar dados sobre seus usuários além de coletar e mapear dados com
uma finalidade específica. O \emph{Machine learning}\footnote{Machine
  Learning: um ``aprendizado de máquina'' que consiste em reconhecimento
  de padrões de dados e informações, captados por computadores e
  utilizados em inteligência artificial (Kosinski,Stillwell e Youyoua ,
  2014)} é o método capaz de identificar, padronizar e fazer julgamentos
de personalidade com base, por exemplo, nas curtidas que uma publicação
ou página receber no \emph{Facebook.} Segundo os autores
Kosinski,Stillwell e Youyoua (2014), essas são as pegadas mais genéricas
e facilmente rastreáveis da internet, outras mais complexas e de maior
capacidade para identificar e qualificar indivíduos são amplamente
utilizadas. Outro exemplo mais complexo do \emph{machine learning} são
as redes de influência que um indivíduo pode ter de acordo com o número
de amigos, curtidas de amigos em uma determinada postagem e
compartilhamentos. Essas análises são utilizados de modo a descrever
e traçar um perfil do usuário do \emph{Facebook}, e servem como
validação através desses amigos e seu nível de influência. Essas pegadas
digitais são, segundo Matza, Navec, Kosinski e Stillwell (2017),
amplamente utilizadas e capitalizadas, uma vez que traçam um perfil
psicológico de seus usuários. O Facebook restringiu recentemente uma
página que traçava perfis psicológicos de seus usuários, ainda que esses
se submetessem aos testes voluntariamente, não possuíam consciência de
como seus perfis estavam sendo utilizados. Contudo, o mesmo perfil pode
ser traçado e capitalizado com as curtidas e os perfis de influência.

Curtir páginas e postagens com um perfil mais introvertido traça um
perfil de introversão assim como um perfil de extroversão pode ser
revelado de forma contrária, e essa informação pode ser útil a quem
deseja estimular esses perfis a consumir algo ou seguir um padrão,
usando abordagens apropriadas para cada perfil. É um método de modulação
que permite abordagens diferentes traçando múltiplos perfis efetivando o
uso da psicologia social em larga escala que convergem em um mesmo
contexto de persuasão em massa (Matza,Navec,Kosinski e Stillwell, 2017).
Portanto, sendo possível classificar pessoas através de suas pegadas
digitais, é igualmente possível manipula-las através desse instrumento.
É um algoritmo que certamente requer uma constante adaptação às mudanças
nos padrões digitais para manter sua precisão, mas sem dúvida muito
eficaz.

Matza, Navec, Kosinski e Stillwell (2014) apontam ainda que o acesso a
websites e testes do Facebook podem ser utilizados para inferir perfis
psicológicos, dividir usuários em segmentos desenhando perfis de
personalidade e mesmo de consumo. O autor ainda cita cinco traços
mensuráveis por websites:

\begin{itemize}
\item
  Abertura a experiências;
\item
  Consciência;
\item
  Extroversão;
\item
  Concordância ou agradabilidade;
\item
  Neuroticismo;
\end{itemize}

Esses perfis psicológicos podem ser muito úteis ao categorizar usuários
de websites, de maneira que o controle atinge diretamente
características pessoais dos indivíduos e ajudam a criar elos de
identidade com o mecanismo de modulação. Se for possível se reconhecer
no mecanismo, mais fácil é a assimilação desses pelos indivíduos.

A perfilação assume características ainda mais preocupantes quando
utiliza recursos de fisionomia para a sua categorização. Um estudo que
já foi muito utilizado em criminologia para ``prevenção'' de possíveis
condutas criminosas, hoje é considerado quase como superstição e até
mesmo racismo. Contudo, Kosinski e Wang (2017) apontam para recentes
progressos em inteligência artificial e visão computadorizada que estão
sendo direcionadas para a adoção de \emph{Deep Neural
Networks}\footnote{Redes neurais profundas (Tradução do texto de
  Kosinski e Wang, 2017).} ou \emph{DNN}. As \emph{DNN,} são como um
grande cérebro capazes de criar e estabelecer padrões de reconhecimento
por meio de imagens. Superam de longe o reconhecimento humano podendo
detectar expressões faciais, padronizar e classificar características,
podendo até mesmo detectar câncer de pele.

Uma discussão recente nos jornais\footnote{Disponível em:
\emph{https://oglobo.globo.com/sociedade/cameras-que-interpretam-expressoes-faciais-causam-polemica-no-metro-de-sao-paulo-23027799}.
  Acesso em setembro de 2018.} abordou o tema com preocupação, posto que
o Metrô de São Paulo foi criticado por captar expressões faciais para
direcionar a publicidade veiculada no trem de acordo com o perfil dos
passageiros. Uma ação civil pública cobra que o metrô encerre a coleta e
armazenamento de dados dos usuários. Todavia, para Kosinski e Wang (2017)
a questão vai além e suas pesquisas revelam ser possível identificar
características que podem tornar seus detentores alvos de perseguição e
estigmatização. A pesquisa concluiu ser possível identificar através de
traços faciais orientação sexual de homens e mulheres. Segundo o autor,
não se trata de estereotipar os pesquisados, mas de encontrar elementos
que caracterizam a homossexualidade como genética e que podem estar em
características físicas, como um queixo mais delicado para homens.

Essa tipificação algorítmica guarda um perigo imenso para os
homossexuais, uma vez que governos já utilizam ferramentas faciais de
coleta de informações para fazer previsões. É um perigo para todos os
indivíduos que podem cair em uma categorização e serem destinados a
perfis constantemente atingidos por perseguições, abusos físicos e
psicológicos de todo tipo. Kosinski e Wang (2017) destacam que o exemplo
dos homossexuais pode ser novamente direcionado a toda uma tipificação
fenotípica marginalizadora e que mesmo perfis de criminosos, pedófilos e
terroristas estão sendo traçados.

A eficácia dessa perfilação tem um caráter perigoso justamente por cada
vez mais categorizar as pessoas, padronizá-las e apresentar
características de infalibilidade. Os perfis categorizados são todos
moduláveis e a modulação de difícil percepção por aqueles que estão
inseridos nessa dinâmica. A contestação dos sistemas de controle é cada
vez mais rara visto que, de uma maneira ou de outra, os indivíduos se
identificam e são partes desse sistema. Se autorregulam e objetivam ser
um modelo de sucesso pessoal e poder dentro dessa dinâmica. O sucesso de
um e o fracasso do outro é previsto e é justamente essa inferiorização
que move os indivíduos em busca de resultados, sendo o sujeito da
modulação vítima e amplificador de seus mecanismos.

\section{Referências}

BACHRACH, Yoram. GRAEPEL, Thore. KOHLI, Pushmeet. KOSINSKI, Michal.
STILLWELL, David. \textbf{Manifestations of user personality in website
choice and behaviour on online social networks}\emph{.} Mach Learn
(2014) 95:357--380.

CORMEN,T.H., LEISERSON, C.E., RIVEST, R.L., STEIN, C.,
\textbf{Introduction to Algorithms,} 3rd edition, MIT Press, 2009.
Rivest, C.
Stein, \emph{http://mitpress.mit.edu/books/introduction-algorithms} {\emph{Introduction
to Algorithms}}, 3rd edition, MIT Press, 2009~DELEUZE, Giles.
\textbf{CONVERSAÇÕES} coleção TRANS. Tradução Peter Pál Pelbart
1972/1990. São Paulo Ed. 34.

DELEUZE, Gilles \& GUATTARI, Félix. \textbf{Mil Platôs: Capitalismo e
Esquizofrenia,} Vol. 5. São Paulo: 34, 1997 (1980).

DELEUZE, Gilles \& GUATTARI, Félix. \textbf{O Anti-Édipo.} São Paulo:
34, 2010 (1972).

DELEUZE, Giles. \textbf{Postscript on the society of control.} October,
Vol. 59. (Winter, 1992), pp. 3-7.

FOUCAULT, Michel.~\textbf{Arqueologia do saber.}~Rio de Janeiro:
Forense, 2008.

FOUCAULT, Michel.~\textbf{Em Defesa da Sociedade}.~São Paulo: Martins
Fontes, 1999.

FOUCAULT, Michel\emph{.} \textbf{Microfísica~do Poder.} 11ª ed., Rio de
Janeiro: Graal, 1997.~

FOUCAULT, Michel.~\textbf{O nascimento da Biopolítica.}~São Paulo:
Martins Fontes, 2008.

GORZ, André. \textbf{O Imaterial:Conhecimento, Valor e Capital}.São
Paulo, Editora Annablume, 2005.

HUI, Yuk. \textbf{Modulation after Control. New formations: a journal of
culture/theory/politics}, Volume 84-85, 2015. pp. 74-91 (Article).
Published by Lawrence \& Wishart.

KOSINSKI, Michal. STILLWELL, David. YOUYOUA, Wu\textbf{. Computer-based
personality judgments are more accurate than those made by
humans.}Department of Psychology, University of Cambridge, Cambridge CB2
3EB, United Kingdom; Department of Computer Science, Stanford
University, Stanford, CA 94305. Edited by David Funder, University of
California, Riverside, CA, and accepted by the Editorial Board December
2, 2014 (received for review September28, 2014)

KOSINSKI, Michal. WANG, Yilun. \textbf{Deep neural networks are more
accurate than humans at detecting sexual orientation from facial
images}\emph{.} Graduate School of Business, Stanford University,
Stanford, CA94305, USA. ©American Psychological Association, 2017.

LARIVIÈRE, Jason. \textbf{Logic of digital worlds. Yuk Hui, on the
existence of digital objects.} University of Minnesota Press, 2016.
Parrhesia 27 • 2017 • 129-135

MATZA, S.C. NAVEC, G. KOSINSKY, M. STILLWELL, D.J. \textbf{Psychological
targeting as an effective approach to digital mass persuasion.} Columbia
Business School, Columbia University, New York City, NY 10027; Graduate
School of Business, Stanford University, Stanford, CA 94305; Wharton
School of Business, University of Pennsylvania, Philadelphia, PA 19104;
Cambridge Judge Business School, University of Cambridge, Cambridge, CB2
3EB, United Kingdom. Edited by Susan T. Fiske, Princeton University,
Princeton, NJ, and approved October 17, 2017 (received for review June
17, 2017).

MARX, Karl. \textbf{O Capital}, Livro I, volume I. São Paulo: Nova
Cultural, 1988.

NEGRI, Antonio; HARDT, Michael.\textbf{~Bem-estar comum}. Rio de
Janeiro: Record, 2016.

RAUTER, Cristina. DE CASTRO PEIXOTO, Paulo de Tarso.
\textbf{Psiquiatria, Saúde Mental e Biopoder: Vida, Controle e Modulação
no Contemporâneo.} Psicologia em Estudo, vol. 14, núm. 2, abril-junho,
2009, pp. 267-275. Universidade Estadual de Maringá. Maringá, Brasil.

ROSE, Nikolas. \textbf{Psicologia como uma Ciência Social.} Psicologia
\& Sociedade, vol. 20, núm. 2, mayo-agosto, 2008, pp. 155-164.
Associação Brasileira de Psicologia Social. Minas Gerais, Brasil.

SIMONDON, Gilbert. \textbf{L'individuation à la lumière des notion de
forme et d'information.} Editions Jérôme Millon, Paris 2005, p91.
Hereafter ILFI.

TRAVERSO-YÉPEZ, Martha. \textbf{Os discursos e a dimensão simbólica: uma
forma de abordagem à Psicologia Social.} Universidade Federal do Rio
Grande do Norte. Estudos de Psicologia 1999, 4(1), 39-59.

TRINDADE, Hélgio (org.). \textbf{O Positivismo: teoria e prática}. 3ª
ed. Porto Alegre: UFRGS, 2007.UHNG HUR, Domenico. \textbf{Da biopolítica
à noopolítica: Contribuições de Deleuze.} LUGAR COMUM No40, pp. 201-215.

Ward, John William. \textbf{"Review of "The Self-Made Man in America:
The Myth of Rags to Riches"} (1954)". Journal of American History. 42
(2).
